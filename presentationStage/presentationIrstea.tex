\chapter{Organisme d’accueil IRSTEA}
\section{Présentation}
\paragraph{}
Établissement publique à caractère scientifique et technologique, crée en 1971, celle-ci change plusieurs fois de dénomination pour devenir \gls{IRSTEA}  en 2012 sous la double tutelle des ministères de l’agriculture et de la recherche scientifique. C’est  un réseau de 19 \gls{UR} et \gls{UMR} répartient en France dans 9 centres dont : Aix-en-Provence, Antony, Bordeaux, Clermont-Ferrand, Grenoble, Lyon-Villeurbanne, Montpellier et Nogent. 
\paragraph{}
L’IRSTEA est un établissement pluridisciplinaire et est tourné vers l’appui aux politiques publiques. Ses activités de recherche et d’expertise impliquent un partenariat fort avec les universités, les écoles et les organismes de recherche français et européens, les acteurs économiques et porteur de politique publique. 

\section{Petite historique}
\paragraph{}	
En 1971, des centres nationaux d’études techniques et de recherches technologiques pour l’agriculture, les forêts et l’équipement rural \gls{CERAFER} furent créés afin de mener des études dans différents domaines, notamment l’agriculture de montagne, le suivi des innovations techniques, ou les problèmes liés à l’utilisation ou à la maîtrise de l’eau. Dès 1973, la dénomination devient Centre technique du génie rural des eaux et des forêts \gls{CTGREF}.
\newline
L’appellation de la structure change et devient \gls{EPA} en 1982 avec la dénomination Centre national du machinisme agricole du génie rural, des eaux et des forêts \gls{CEMAGREF}. Elle se transforme en établissement public à caractère scientifique et technologique \gls{EPST} en 1986 sous la double tutelle des ministères chargés de l’agriculture et de la recherche. C’est en février 2012 que la structure est nommée IRSTEA, Institut national de recherche en sciences et technologies pour l’environnement et l’agriculture

\section{Unité de recherche TSCF}
\paragraph{}
L’unité de recherche \gls{TSCF}, composée de 3 équipes qui rassemblent 60 agents, est implantée sur 2 sites : le Pôle scientifique et universitaire des Cézeaux à Aubière (63) et le Site de recherche et d’expérimentation de Montoldre (03). 

Elle mobilise les sciences pour l’ingénieur et les sciences et technologies de l’information et de la communication pour conduire des recherches sur les méthodes et outils pour une ingénierie des systèmes agro-environnementaux. 
\newline
Elle conduit également des activités de recherche, d’expertise et d’essai dans le domaine de la sécurité et des performances des agroéquipements pour contribuer à l’amélioration de la sécurité en agriculture et à la réduction des pollutions d’origine agricole. 
\paragraph{}
Grâce à ses travaux de recherche technologique, elle apporte des réponses concrètes aux besoins d’une agriculture productive écologiquement responsable et de la gestion de l’environnement. Ses activités relèvent du département Écotechnologies d'IRSTEA 
\newline
Fortement ancrée dans la dynamique régionale de recherche et d’innovation, l’Unité de recherche TSCF est membre du Laboratoire d'Excellence. 

\section{Equipe \gls{COPAIN}}
\paragraph{}
L’activité de l’équipe, basée à Aubière et à Montoldre, est consacrée aux méthodes d’ingénierie des systèmes d’information communicants dédiées à la gestion agro-environnementale. Cet ensemble de méthodes couvre l’analyse des besoins des acteurs, la spécification des systèmes d’information, leur modélisation, leur conception, leur gestion et leur lien avec les sources de données. 
\paragraph{}
Les chercheurs de COPAIN sont spécialisés en informatique et dans les systèmes d'information, avec une solide expérience de projets interdisciplinaires. L'activité de cette équipe est dédiée aux méthodes d'ingénierie de systèmes d'information pour la gestion agro-environnementale. Ces méthodes couvrent les besoins des acteurs, la définition des caractéristiques des systèmes d'information, leur modélisation, leur gestion. 
\newline
Le but de l'équipe est de développer des méthodes et des techniques pour : 
\begin{itemize}
    \item Déployer et gérer des réseaux de capteurs sans fil adaptés aux problèmes agro-environnementaux. 
    \item Concevoir et gérer des systèmes d'information, tels que des entrepôts de données ou des systèmes de gestion de connaissance, adaptés au contexte agro-environnemental. 
\end{itemize}