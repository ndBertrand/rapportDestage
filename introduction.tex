\renewcommand{\partname}{}
\renewcommand{\chaptername}{}
\renewcommand{\thechapter}{}
\renewcommand{\thesection}{}



\chapter{Introduction}
\paragraph{}
Ce stage s’inscrit dans le cadre de ma formation en Master I Informatique à l’Institut d’Informatique de l’université Clermont Auvergne. Par ce stage, de plus de 5 mois, j’ai eu l’occasion d'enrichir mes capacités d’analyse et de réflexion en appliquant les méthodologies apprisent au cours de mon cursus. 
Au sein de l’Irstea, j’ai intégré l’équipe COPAIN spécialisée dans la gestion et le déploiement de réseaux de capteurs, la conception des systèmes d’information et des systèmes de gestion de connaissance adaptés au contexte agro-environnemental.
\paragraph{}
Chaque jour, le numérique s’invite davantage dans le quotidien des agriculteurs et s’inscrit dans une nouvelle narration de l’innovation en agriculture. il offre des perspectives inédites pour la prédiction et l’aide à la  décision dans la production végétale. Nombreuses sont les applications qui peuvent en découler: le suivi des conditions météorologique, l’analyse de l’état hydrique des cultures et du sol, l’estimation du rendement et la qualité des cultures sont des exemples de l'efficacité de l’outils numérique. Mais cette efficacité ne peut être rendu possible que s’il existe en amont un environnement facilitant les échanges des données. Pour faciliter l’accès et la valorisation de ces données, il faut faire émerger de nouveaux services dans une chaîne de confiance dans un écosystème de gestion des consentements. l’accès au données des agricultures doit pouvoir se faire en toute transparence et conditionné à une bonne compréhension de l’utilisation.
\paragraph{}
la première mission de mon stage fut donc de faire une étude sur l’un des aspect  de la protection des données  de l’écosystème de gestion des consentements du projet MULTIPASS qui est l’anonymisation. Cette étude pourra constitué une base pour une étude plus approfondi sur la confidentialité et la protection des données.
L’autre mission était de proposer une solution de d’acquisition et de stockage de données issus du site expérimental Irstea de Montoldre dans le cadre du projet \gls{OPEROSE}. Jusqu’a présent, l'Irstea, et surtout l’équipe COPAIN, disposait de plusieurs systèmes de stockage indépendants adaptés à répondre aux besoins pour lesquelles ils avaient été crée. L’équipe COPAIN souhaitait avoir un système de stockage centralisé des données capteurs.

\paragraph{}
Ce rapport résume mon travaille durant la période de stage. Le document est ainsi subdivisé en 4 parties, la première partie présente le contexte du stage, la deuxième et la troisième partie reviennent respectivement sur les projet auxquels j’ai participé et la quatrième partie fait un bilan du stage.


