% Keywords command
\providecommand{\keywords}[1]
{
  \small    
  \textbf{\textit{Keywords---}} #1
}


\selectlanguage{french} 
\begin{abstract}
\paragraph{}
Depuis 2008, le gouvernement, au travers des plan Ecophyto I et Ecophyto II, a mis en place une politique visant à réduire progressivement l’utilisation des produits phytosanitaire (communément appelés pesticides) tout en maintenant une agriculture performante. Ces plans ont mené en 2017 à l’élaboration du Challenge Rose (Robotique et Capteur au Service d’Ecophyto) encourageant le développement de technologies et d’outils en lien avec l’agriculture numérique. 
\paragraph{}
Mon travail, s'inscrit dans la continuité de la mise en œuvre de ce projet en proposant une solution de stockage de données capteurs issus du site expérimental \gls{IRSTEA} de Montoldre. Ainsi donc, dans ce rapport de stage, sont présentés les différentes phases d’analyse, de conception et de développement de ladite solution. 
\paragraph{}
Ce rapport présente également une étude de sur l’anonymisation des données dans le cadre du projet CASDAR MULTIPASS. En effet, avec l’émergence du numérique, les exploitations agricoles sont une source de données incontournable. Ainsi, ce travaille fait un état de l’art sur les techniques existantes de l’anonymisation des données afin de les transposées dans le domaine agricole pour créer un écosystème de gestion des consentements des agriculteurs protégeant les échanges de données des exploitations. 
\paragraph{}
\keywords{capteurs, base de données générique, anonymisation}
\end{abstract}


\selectlanguage{english} 
\begin{abstract}
 Since 2008, the government, through the Ecophyto I and Ecophyto II plans, has put in place a policy aimed at progressively reducing the use of plant protection products (commonly known as pesticides) while maintaining efficient farming. In 2017, these plans led to the development of the Rose Challenge (Robotics and Sensors in the service of Ecophyto) encouraging the development of technologies and tools related to digital agriculture.
\paragraph{}
My work is part of the continuity of the implementation of this project by proposing a storage solution for sensor data from the IRSTEA experimental site in Montoldre. Thus, in this internship report, are presented the different phases of analysis, design and development of said solution.
\paragraph{}
This report also presents a study on the anonymization of data within the CASDAR MULTIPASS project. Indeed, with the emergence of digital, farms are an essential source of data. Thus, this work makes a state of the art on the existing techniques of the anonymization of the data for the transposed in the agricultural field to create an ecosystem of management of the consents of the farmers protecting the exchanges of data of the exploitations.
\paragraph{}
\keywords{sensors, generic database, anonymisation}
\end{abstract}
   