\section{Conclusion}
le but de ce stage était premièrement le développent d'une solution pour l'acquisition et le stockage des données capteurs du site IRSTEA de Montoldre et deuxièmement faire une étude sur l'anonymisation des données et tester les différentes technique d'anonymisation sur des jeux de données.

Le résultat de ces deux projet sont respectivement un système d'acquisition automatique, une base de données capteurs, un système de visualisation et un rapport sur l'anonymisation des données. Le développement de la solution d'acquisition et stockage des données capteurs permettra aux participants du Challenge ANR ROSE d'avoir un ensemble d'informations qui leur permettra de faire une bonne interprétations de données issus du site de Montoldre. L'étude sur l'anonymisation des données constituera une base pour faire une étude plus approfondit sur un système de gestion de consentement dans le partage des données issus des exploitations agricole.
Ce stage fut une expérience très satisfaisante et enrichissante. Faire partie d'une équipe accueillante et dynamique, travailler dans des bonne conditions et pouvoir mettre en pratique le savoir que j'ai acquis sont autant de bonne choses que j'en retire.
J'ai pu, grâce à ce stage, améliorer mes compétences et élargir mes connaissance en ce qui concerne le traitement et l'analyse de grands volumes de  données.
