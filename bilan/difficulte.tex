\section{Difficulté rencontrées}
la difficulté principale fut l'adaptation aux différents outils utilisés. En effet, un temps non négligeable fut consacré à l'apprentissage d'un tel ou tel autre langage de programmation d'une telle librairie etc . L'obstacle ne résidait pas dans la façon de faire les choses, mais plutôt dans la façon de les faire dans un contexte bien précis. En d'autres mot la difficulté était plus du côté technique. Cela m'a permis de voir mes forces et mes faiblesse, et m'a surtout encouragé à me surpasser, en recherchant toujours des solutions appropriées et adéquates. 

Un exemple pour illustrer les propos ci-dessus est qu'en essayant de me familiariser avec l'API sencrop pour le projet OPEROSE, j'ai remarqué que la documentation n'était tout a fait au point, ce qui est normale car leur API est aussi toujours en cours de développement. Le manque d'une documentation complète m'a fait faire des centaines de requête pour afin comprendre comment est structuré l'API et déceler au se trouve les données que l'on désire avoir. Cette petite expérience m'a témoigné l'importance d'une bonne documentation, qui je l'espère, fera partie de tous mes futures projets.
