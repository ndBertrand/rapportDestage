\section{Difficultées rencontrées}
La difficulté principale fut l'adaptation aux différents outils utilisés. En effet, un temps non négligeable fut consacré à l'apprentissage des langages de programmation et de librairies utilisés pour le développement de l'outils. L'obstacle ne résidait pas dans la façon de faire les choses, mais plutôt dans la façon de les faire dans un contexte bien précis. En d'autres mots la difficulté était plus du côté technique. Cela m'a permis de voir mes forces et mes faiblesse, et m'a surtout encouragé à me surpasser, en recherchant toujours des solutions appropriées et adéquates. 

Un exemple pour illustrer les propos ci-dessus : en essayant de me familiariser avec l'API sencrop pour le projet OPEROSE, j'ai remarqué que la documentation n'était tout a fait au point, ce qui est normale car leur API est aussi toujours en cours de développement. Le manque de documentation complète m'a obligé à réaliser un grand nombre de requêtes  pour comprendre  structuration de l'API et déceler où se trouvent les données que l'on désire obtenir. Cette petite expérience m'a démontré l'importance d'une bonne documentation, qui je l'espère, fera partie de tous mes futurs projets.
