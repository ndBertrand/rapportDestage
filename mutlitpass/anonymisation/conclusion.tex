\chapter{Conclusion}
Compte tenu de la réglementation actuelle, il y a obligation d'assurer une  confidentialité minimale  des données en général et des données à caractères personnelles en particulier. Il existe plusieurs méthodes et techniques qui assurent cette confidentialité mais le choix de celles-ci dépendra de la sensibilité des données ou des applications visées. Ce choix dépendra aussi de l'ensemble des moyens (techniques et financiers) raisonnablement susceptible d'être utilisé pour identifier une personne physique directement ou indirectement. Bien que le domaine agricole soit en retard par rapport au monde de la santé et de la finance en matière de protection des données, il est essentiel que celui-ci puisse vite se développer pour renforcer la confiance des producteurs nécessaire aux partages de leurs données et permettre ainsi de faire émerger de nouvelles connaissances et de nouveaux services.