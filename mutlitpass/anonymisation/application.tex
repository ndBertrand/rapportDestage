\chapter{Application des techniques d'anonymisation dans le domaine agricole}
Jusqu'à présent, rares sont les études qui portent sur l'anonymisation des données dans le domaine agronomique.
Cela est en partie dû au fait que ces données sont en générale ouvertes et ne portent pas un caractère personnel. On peut facilement trouver,sur des plate-formes comme opendata ou datagouv, de grandes volume de données en rapport avec l'agro-environnement. Cela est certes indispensable pour mener à bien des études comme par exemple le rapport entre le climat et l'apparition de certaines maladies des plantes, l'influences des pesticides sur l'environnement etc. 
Pour faire cela, on est souvent amené à croisé des données du type plante cultivées, type de sol avec des données météorologiques localisé se rapportant à une parcelle de terrain. L'ensemble des parcelles cultivés en France étant répertoriés dans le \gls{RPG}, il apparaît clairement que croisé avec d'autres sources de données, le RPG peut permettre une ré-identification des agricultures, sources de données, d'où l'intérêt d'anonymiser également les données dans le domaine agronomique.


\subsection{Exemple de réidentification}
En parcourant un jeu de données sur le Plan régional de Modernisation des Bâtiments d’Élevage "classique" en Auvergne (PMBE)- Attributions 2013/2015  de la plateforme opendata, j'ai pu isoler un enregistrement \ref{fig:Enregistrement dans base de données publique}
\begin{figure}
    \centering
    \includegraphics[width=\textwidth]{images/anonymisation/exemple1_anonymisation.png}
    \caption{Enregistrement dans base de données publique}
    \label{fig:Enregistrement dans base de données publique}
\end{figure}
Après quelque recherche sur un moteur de recherche, en utilisant les mots clés poulailler, Allier, 1800\begin{math}m^{2}\end{math} j'ai trouver deux articles de journal qui en parler et à partir de là j'ai pu identifier la personne propriétaire du poulailler.
\begin {figure}
\begin{center}
    \hbox{ 
    \includegraphics[height=3cm]{images/anonymisation/exemple2_anonymisation.png}
    \hspace*{1cm}  %% pour mettre un espace (horizontal) de 5cm entre les deux images
    \includegraphics[height=3cm]{images/anonymisation/exemple3_anonymisation.png}
  }
\caption{Articles de Journal}
\label{fig : Articles de Journal}
\end{center}
\end {figure}

Dans l'exemple précèdent, on remarque à quel point il peut être parfois facile de ré-identifier une personne à partir de données publiques. Dans la suite de ce rapport, j'essaye de transposé  les techniques d'anonymisation vues précédemment sur des données à caractères agronomique en se focalisant surtout sur la dimension spatiale des données car la dimension temporelle peut être facilement masquer. Le but étant de conserver l'utilité des données.

%%%%%%%%%%% GÉNÉRALISATION %%%%%%%%%%%
\section{La Généralisation}
L' approche la plus couramment utilisée pour rendre anonyme des données géographique est la généralisation. Elle se rapporte à la technique de généralisation vue plus haut. Elle consiste à réduire la précision au niveau géographique. On crée ainsi de classe d’équivalence, avec des enregistrements ayant une localisation spatiale identique. Selon la clarté sur les données et le niveau de sécurité que l’on souhaite, la réduction de la précision géographique peut être utilisé pour contrôler la taille des classes d’équivalence. Deux techniques ressortent parmi les techniques de généralisation:  
\begin{itemize}
    \item Le Changement d’échelle : cette approche consiste à réduire la précision géographique en élargissant les zones visées par les attributs géographiques. Le changement d'échelle peut s'obtenir de plusieurs façon : \begin{itemize}
        \item  On peut supprimer les trois derniers chiffres des codes postaux, se référent à une grande zone géographique.
        \item   Les coordonnées GPS peuvent être remplacé par le nom de la commune, de la région ou du canton.
    \end{itemize} 
    Le changement d’échelle est donc tout simplement une application du k-anonymat sur des données géographiques. 
    \item   Le Carroyage : Le carroyage est une technique de quadrillage utilisée en topographie, afin de rassembler et de traiter des données en vue d’une exploitation cartographique ou statistique. Il est très similaire au changement d’échelle à l’exception que pour celui-ci, toutes les subdivisions sont de même taille ce qui n’est pas le cas pour changement d’échelle. 
\end{itemize}

%%%%%%%%%%% Génération des données %%%%%%%%%%%
\section{La Génération des données}
Une autre approche serait la génération des données. C'est-à-dire qu'a partir d'un certain jeu de données, on pourrait en générer un autre en ajoutant ou en supprimant de l'information. Cela reviendrait à appliquer la technique de l'ajout de bruit.Il resterait alors à surmonter la difficulté de rendre cohérent les données ainsi générées. En effet, Comme vu précédemment, l'ajout de bruit est souvent confronter à deux problèmes : soit le bruit n'est pas suffisant et dans ce cas l'anonymisation n'est pas efficace, soit il y a beaucoup de bruit et la données perd son utilité.
%%%%%%%%%%% Déplacement des Coordonnées %%%%%%%%%%%
\section{Déplacement des coordonnées}
 La dernière approche pour qui peut être utiliser sur des coordonnées spatiales consisterait à faire une permutation(brouillage) des coordonnées. Il faudra cependant respecter certaines contraintes pour garder la cohérence des données ainsi que leur utilité. Exemples : 
 \begin{itemize}
     \item deux points de coordonnées proches avant le brouillage, doivent l'être aussi après le brouillage
     \item brouiller les coordonnées de façon uniforme dans chaque zone Figure \ref{fig:Brouillage uniforme}
     \item Mais si par exemple la station météo la plus proche d’une parcelle de la zone A se trouve dans la zone B, alors brouiller les coordonnées en tenant compte du voisinage. Figure \ref{fig:Brouillage avec gestion de voisinage}
 \end{itemize}
 
 \begin{figure}[!h]
    \centering
    \includegraphics[width=.5\textwidth]{images/anonymisation/brouillage_image1.png}
    \caption{ brouillage uniforme}
    \label{fig:Brouillage uniforme}
\end{figure}
\begin{figure}[!h]
    \centering
    \includegraphics[width=.5\textwidth]{images/anonymisation/brouillage_image2.png}
    \caption{ Brouillage avec gestion de voisinage}
    \label{fig:Brouillage avec gestion de voisinage}
\end{figure}
