\subsection{T-proximité}
La t-proximité est un raffinement supplémentaire de la l-diversité. Elle est utilisée pour préserver la confidentialité d'un ensemble de données en réduisant la granularité de la représentation de celui-ci. Cette réduction est un compromis qui entraîne une certaine perte d'efficacité des algorithmes de gestion des données ou d'exploration afin de gagner en confidentialité. Elle traite les valeurs d'un attributs en prenant en compte de la distribution des valeurs pour cet attribut.
La t-proximité formalise donc l’idée d'une connaissance globale en exigeant que la distribution d’un attribut sensible soit, pour toute classe d'équivalence, proche de sa distribution dans l’ensemble du jeu de données. 
\subsubsection{Principe de la t-proximité}
On dit qu'une classe d'équivalence a une proximité t si la la distance entre la distribution d'une attribut sensible dans cette classe et la distribution de l'attribut la la table entière n'est pas supérieure à un seuil t. Une table a une proximité t si toutes les classes d'équivalence ont la proximité t.