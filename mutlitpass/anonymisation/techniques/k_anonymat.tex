\subsection{K-anonymat} 

La k-anonymisation est une technique d’anonymisation qui empêche qu’une personne ne soit isolée dans un jeu de données. Elle permet ainsi de regrouper un individu avec au moins k autres individus, de sorte qu’il y ait moins de chance de le retrouver. En d’autres mots, on généralise certains attributs de telle manière à ce que ces derniers soient identiques pour un groupe de K individus\cite{noauthor_institut_1992}. 

\paragraph{Etapes de la k-anonymisation:} 
\begin{itemize}
    \item  Déterminer les ensembles d’attributs qui peuvent être utilisés pour croiser les données anonymes avec des données identifiants. 

    \item Réduire le niveau de détail des données de telle sorte qu’il y ait au moins k individus qui ont la même valeur de quasi-identifiants. 
\end{itemize}

%% ajout d'exemple
\begin{figure}
    \centering
     
    \includegraphics[width=1\textwidth]{images/anonymisation/k_anonym_image1.png}
    \caption{k-anonymisation données 4-anonymes}
    \label{k-anonymisation données 4-anonymes}
\end{figure}


\paragraph{Evaluation du k-anonymat:}  
\begin{itemize}
    \item \textbf{Individualisation :} étant données les classes d’équivalence, nous savons qu’au moins k individus partagent certains attributs dans le jeu de données. Il ne devrait donc plus être possible d’isoler un individu dans un groupe de k individus. 

    \item \textbf{Corrélation:} il est possible de relier les enregistrements par groupe de k-individus. Au sein d’un même groupe, la probabilité que deux enregistrements correspondent à un individu est de 1/k. 

    \item \textbf{Inférence:} si tous les k individus d’une classe d’équivalence ont la même valeur pour un attribut donné et que cette information est sensible, il suffit de connaître à quelle classe d’équivalence appartient un individu pour déduire sa donnée sensible. 
\end{itemize}
 
Alors que k-anonymat protège contre la divulgation d'identité, il n'offre pas une protection suffisante contre la divulgation d'attributs. Cela a été reconnu par plusieurs auteurs\cite{truta_privacy_2006,machanavajjhala_l-diversity:_2006}. Deux types d'attaques ont été identifiées pour le k-anonymat.
\begin{itemize}
    \item attaque par homogénéité \textbf{Homogeneity Attack} : Tous les individus d'une classe d'équivalence possèdent la valeur pour un attribut sensible
    \item  attaque avec une connaissance de base \textbf{Background Knowledge Attack} \cite{machanavajjhala_l-diversity:_2006} : si l'attaquant connaît déjà certaines informations sur un individu, il sera facile pour lui de l'isoler dans une classe de k-individus. 
\end{itemize} 
