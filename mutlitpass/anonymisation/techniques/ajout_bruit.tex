\subsection{L’ajout de Bruit} 

L’ajout de bruit consiste à rendre moins précis un ensemble de données en ajoutant ou en retirant de l’information sur les enregistrements tout en conservant la distribution générale. Les enregistrements restent identifiables mais sont moins fiables [15]. 

L’ajout de bruit doit normalement être combiné à d’autres techniques d’anonymisation pour être efficace\cite{noauthor_repertoire_2018-1}.  

\paragraph{Évaluation de l’ajout de bruit : }

\begin{itemize}
    \item  \textbf{Individualisation}: l’ajout de bruit n’empêche pas la possibilité d’une isolation. Même si l’enregistrement est moins fiable, il reste possible d’isoler tous les enregistrements concernant un individu. 
    
    \item  \textbf{Corrélation}: il est toujours possible de relier les enregistrements (issu d’un ou de plusieurs jeux de données différents) d’un même individu. 
    
    \item \textbf{Inférence}: le taux de succès avec une attaque par inférence est beaucoup moins élevé. 
\end{itemize}

\paragraph{Erreurs courantes}:   
\begin{itemize}
    \item Ajout de bruit incohérent : si le bruit ne respecte pas la logique des attributs ou si celui-ci est disproportionné, on peut alors imaginer qu'un attaquant puisse identifier le bruit, le filtrer et ainsi recréer un jeu de données. 
    \item supposer que l’ajout de bruit est suffisant : L'ajout de bruit n'est pas une technique qui se suffit. Pour être efficace, elle doit s'utiliser en complément avec d'autres techniques pour rendre difficile la récupération des données.
\end{itemize}

