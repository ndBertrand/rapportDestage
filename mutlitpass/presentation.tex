\section{Présentation du projet} 
L'agriculture numérique, fait que les exploitations agricoles sont aujourd’hui une source immense de données. De ce fait, les institutions techniques agricoles ont formulé dix recommandations pour faciliter l’accès  et la valorisation de ces données. Ainsi à travers un financement CASDAR,  le ministère en charge de l’agriculture lance en 2017 le projet  MULTIPASS. Ce dernier vise à faire émerger de nouveaux services pour l’agriculteur dans une chaine de confiance gérant les consentements d’accès aux données des exploitations. Le projet renforcera ainsi la confiance nécessaire au partage de données et permettra d’apporteR une solution aux questions des agriculteurs sur la maîtrise de leurs données et la transparence des usages qui en sont faits. 

Plus précisément, les objectifs du projet sont les suivants : 

\begin{itemize}
    \item Proposer un écosystème de gestion des consentements interopérable entre les acteurs qui apporte confiance, simplification et sécurité aux producteurs et valorisateurs de données. 
    \item Favoriser l’innovation ouverte, c’est-à-dire l’émergence d’applications agronomiques couplées aux données des agriculteurs provenant de n’importe quelle source de données ou objet connecté, pour éviter le risque de concentration de l’innovation. 
    \item Favoriser la création de connaissances par l’analyse de données massives d’exploitations, dans une chaîne de confiance. 
\end{itemize}

Partenaires : ARVALIS (Coordinateur), ACTA, FIEA, IDELE, Irstea, Orange, Smag.
